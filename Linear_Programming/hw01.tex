\documentclass[UTF8]{ctexart} 
\usepackage{amsmath}
\usepackage{booktabs}
\usepackage[a4paper,left=1.25in,right=1in, top=1in,bottom=1in]{geometry}
\ctexset{section={format={\Large \bfseries}}}

\pagestyle{plain}
\title{Homework01}
\author{Amomorning}
\date{\today}

\begin{document}
\maketitle

\section{单纯形法}
\paragraph{题目:} \kaishu{用单纯形法求解如下LP问题并结合计算机实现。}
\begin{gather*}
\max_{x_1,x_2, x_3}\quad Z = 5x_1 + 8x_2 +6x_3 \\
\begin{aligned}
\textup{s.t.}\quad x_1 + x_2 + x_3  &\leq  12\\
                   x_1 + 2x_2 + 2x_3 &\leq  20 \\
                   x_1, x_2, x_3 &\geq 0 \\
\end{aligned}
\end{gather*}

\subsection{转化为标准型}
asdasd



\section{建模并求解}

\paragraph{题目:} \kaishu{某厂按合同规定须于当年每个季度末分别提供10、15、25、20台同一规格的柴油机。已知该厂各季度的生产能力及生产每台柴油机的成本如表所示。如果生产出来的柴油机当季不交货的话,每台每积压一个季度需储存、维护等费用1500元。要求在完成合同的情况下,做出使该厂全年生产费用最小的决策。}

\begin{table}[!htpb]
    \centering 
    \begin{tabular}{ccc}
    \toprule
    \kaishu{季度} & \kaishu{生产能力(台)} & \kaishu{单位成本(万元)} \\
    \midrule
    1 & 25 & 10.8 \\
    2 & 35 & 11.1 \\
    3 & 30 & 11.0 \\
    4 & 10 & 11.3 \\
    \bottomrule
    \end{tabular}
\end{table}

\subsection{}

\section{对偶理论}
\paragraph{题目:} \kaishu{写出如下问题的对偶形式,并用对偶单纯形法求解。}

\begin{gather*}
    \max_{x_1, x_2, x_3}\quad Z = 5x_1 + 8x_2+6x_3\\
    \begin{aligned}
    \textup{s.t.}\quad  x_1 + x_2 + x_3 &\leq 12 \\
                        x_1 + 2x_2 + 2x_3 &\leq 20 \\
                        x_1 - 4x_2 + 3x_3 &= 15 \\
                        x_1, x_2, x_3 &\geq 0 \\
    \end{aligned}
\end{gather*}

\subsection{}

\end{document}

